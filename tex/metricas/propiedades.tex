\subsection{Propiedades utilizadas para las métricas}

Para las distintas pruebas cuantitativas realizadas, se establecieron tamaños fijos según el tipo de contenido evaluado.

\subsubsection{Sitio Web Estático}

El sitio web estático se clasificó por el tamaño total del contenido alojado:

\begin{itemize}
    \item \textit{1KiB-50files}: 50 archivos que en conjunto tienen un tamaño de 1KiB
    \item \textit{25KiB-50files}: 50 archivos que en conjunto tienen un tamaño de 25KiB
    \item \textit{50KiB-1files}: 1 archivo que en conjunto tienen un tamaño de 50KiB
    \item \textit{50KiB-10files}: 10 archivos que en conjunto tienen un tamaño de 10KiB
    \item \textit{50KiB-25files}: 25 archivos que en conjunto tienen un tamaño de 25KiB
    \item \textit{50KiB-50files}: 50 archivos que en conjunto tienen un tamaño de 50KiB    
\end{itemize}

\subsubsection{Repositorio de conocimiento}

Los artículos se separaron en tres clasificaciones, según su longitud en bytes:

\begin{itemize}
    \item \textit{short}: artículo de 5000 bytes.
    \item \textit{medium}: artículo de 10000 bytes.
    \item \textit{large}: artículo de 50000 bytes.
\end{itemize}

\subsubsection{Mensajero en tiempo real}

Para la medición del envío y recepción de mensajes, se consideró tres tipos de mensajes:

\begin{itemize}
    \item \textit{short}: mensaje de 1 palabra. ''\textit{Lorem}''
    \item \textit{medium}: mensaje de 10 palabras. ''\textit{Lorem ipsum dolor sit amet, consectetur adipiscing elit. Aliquam elementum.}''
    \item \textit{large}: mensaje de 30 palabras. ''\textit{Lorem ipsum dolor sit amet, consectetur adipiscing elit. Aliquam ex risus, porttitor sed lacus id, egestas lobortis purus. Curabitur consectetur metus ut est vehicula egestas. Class aptent taciti sociosqu ad.}''
\end{itemize}

El tamaño de las palabras elegidas representa el largo de los mensajes más comunes que suelen enviarse en un chat.

A partir de los datos obtenidos se calcularon el máximo (\textbf{Max}), mínimo (\textbf{Min}), la media (\textbf{Mean}), el desvío estándar (\textbf{Std}) y la mediana (\textbf{Median}) de cada muestra.