\section{Experimentación y/o validación}

\subsection{Costos}
¿Cuánto nos cuesta desplegar y mantener un servicio en cada ecosistema?

\subsubsection{IPFS}

\subsubsection{Blockchain}


De los casos de uso esperamos responder las siguientes incógnitas: %obtener las siguientes métricas:

\paragraph{Swarm}
Al deployar el sitio web es necesario contar con \textit{postage stamps} que son la manera de pagar por el uso del almacenamiento en Swarm. Cada actualización que se realice al sitio requiere de \textit{postage stamps} y, además, estos tienen fecha de vencimiento por lo que es necesario volver a pagar frecuentemente. Hay que tener en cuenta que dichos \textit{postage stamps} se pagan en la criptomoneda BZZ que fluctúa de valor con respecto al dólar estadounidense.

La obtención del sitio web no requiere de costo alguno, por lo que desde el punto de vista de un usuario de la aplicación no sería necesario pagar.

<TODO: medir cuánto es el costo aproximado en USD o BZZ>

\paragraph{Ethereum}
Se utiliza la moneda ETH para pagar por el despliegue de cada transacción, esto incluye tanto el despliegue de cada \textit{smart contract} como también la edición de un artículo (en el caso del repositorio de conocimiento). Por lo tanto, el usuario final de la aplicación termina pagando por creación y edición de cada artículo en el repositorio de conocimiento, y por cada mensaje enviado en el mensajero en tiempo real. Por otro lado, para las operaciones de lectura no se tiene que pagar nada.

<TODO: medir cuánto es el costo aproximado en USD o ETH>

\subsection{Experiencia de desarrollo} % Developer Experience

¿Qué tan fácil es desplegar en cada ecosistema?

\subsubsection{IPFS}

\subsubsection{Blockchain}

\paragraph{Swarm}
En Swarm existe la herramienta de terminal \href{https://github.com/ethersphere/swarm-cli}{swarm-cli} con la cual se puede interactuar con un nodo de Swarm. También el equipo de Swarm provee una Github Action que permite la posibilidad de automatizar el despliegue generando un pipeline que utilice dicha herramienta.

En cuanto a un ambiente de pruebas o staging, si bien no existe un \textit{gateway} público que interactúe con la \textit{testnet}, es posible levantar uno propio que sí lo haga apuntando a la \textit{testnet} de Sepolia usando la herramienta \href{https://github.com/ethersphere/gateway-proxy}{gateway-proxy}.

\paragraph{Ethereum}
Con la librería web3.js se puede interactuar con un nodo de Ethereum y realizar un despliegue de la aplicación. Además, con las herramientas de Hardhat se puede levantar una red de prueba que facilita el desarrollo local.

\subsection{Viabilidad} 

¿Que tan viable es crear una aplicación comunitaria para cada uno de estos ecosistemas?

\subsubsection{IPFS}

\subsubsection{Blockchain}

\paragraph{Swarm}
Resulta más conveniente para sitios web o recursos estáticos. No es posible la ejecución de código.

\paragraph{Ethereum}
Su punto fuerte es la ejecución de código, por lo cual es útil para funcionar como backend para aplicaciones web. Por el costo de almacenamiento de los smart contracts no es recomendable para sitios o recursos estáticos como imágenes o videos.

\subsection{Performance}

\subsubsection{IPFS}

\subsubsection{Blockchain}

\subsection{Resumen}

% Poner la tabla, hay una en el notion 