\section{Lecciones aprendidas}

\subsection{Tecnologías emergentes}

Trabajar con tecnologías emergentes resulta un desafío al encontrarse en desarrollo constante y frecuente. Esto quiere decir que la documentación es escasa, nula o se encuentra desactualizada. Las bibliotecas pueden cambiar su comportamiento entre versiones, incorporar nuevas funcionalidades inestables o incluso presentar errores no detectados previamente.

Estas dificultades técnicas nos llevaron a interactuar directamente con los proyectos involucrados, revisar el código fuente y abrir \textit{issues} en los repositorios oficiales, detallando los problemas, proponiendo ejemplos mínimos reproducibles y, en algunos casos, sugiriendo soluciones. Esta experiencia, aunque desafiante, resultó sumamente formativa.

Nos permitió entender de forma práctica cómo funciona la colaboración en proyectos de software libre. Aprendimos a comunicarnos de manera clara y efectiva con los mantenedores, a estructurar nuestros aportes para que fueran útiles a otros usuarios y a seguir los canales y convenciones propios del ecosistema \textit{open source}. En varios casos, nuestras contribuciones derivaron en mejoras directas en las bibliotecas que utilizamos.

\subsection{Creación de paquetes}
Durante parte del desarrollo del repositorio de conocimiento, trabajamos la implementación de cada ecosistema dentro del front-end. Esta decisión nos pareció razonable luego de elegir usar un mismo front-end para ambos ecosistemas y abstraernos en una interfaz en común. Sin embargo, una vez avanzado en ambos ecosistemas, las limitaciones de esta elección se hicieron aparentes.

Por un lado, nos restringía para realizar métricas y tests manuales fácilmente, ya que cada prueba involucraba levantar un servidor local con la página y recompilar el mismo con cada cambio, además de no tener fácil acceso para pruebas automáticas.

Por otro lado, limitaba el potencial de las herramientas que estábamos desarrollando. El resultado final es mucho más versátil ya que puede acoplarse a cualquier front-end como se demostro con los front-ends alternativos, ventaja que se alinea con la filosofía de aplicaciones decentralizadas y comunitarias.

En retrospectiva desarrollar los casos de uso de forma modular desde el comienzo hubiera agilizado el proceso, e implicaría posiblemente el uso de pruebas en una etapa más temprana.

\subsection{Otras lecciones}

\begin{itemize}
    \item Paralelizar las tareas por ecosistema nos ayudó a tener un referente para cada uno y no abarcar la misma información al investigar.

    \item Realizar reuniones semanales nos sirvió para compartir el conocimiento adquirido de los distintos ecosistemas y ponernos al tanto de en qué estaba trabajando cada uno.

    \item Nos abrió al mundo P2P y web descentralizada, que es una manera distinta de pensar y desarrollar aplicaciones.

\end{itemize}