\subsection{Caso de uso 1: Sitio web informativo}

Este es el caso más simple, donde no se requiere ningún tipo de procesamiento. El contenido que tiene este sitio es información sobre los casos de uso implementados, como también instrucciones o referencias de sus ecosistemas de despliegue. 

\subsubsection{Requisitos funcionales}

\begin{itemize}
    \item \textbf{Landing page del proyecto}: sitio web informativo donde se presenta lo que se fue haciendo en este proyecto, información sobre los casos de uso y sus ecosistemas de despliegue.
\end{itemize}

\subsubsection{Implementación en IPFS}

...

\subsubsection{Implementación en Blockchain}

Se utilizó el sistema de archivos distribuido llamado \textit{Ethereum Swarm}\cite{ethereum-swarm}. El mismo funciona de manera similar a IPFS pero utilizando una sidechain de Ethereum para almacenar los archivos.

Por esto último, se hace uso de una moneda (BZZ) que funciona a modo de incentivo monetario para que las personas compartan sus recursos de almacenamiento levantando nodos y, de esta manera, lograr que la red se mantenga viva.