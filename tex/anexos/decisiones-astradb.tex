\subsection{Selección del protocolo de transporte para conexiones web en IPFS}

Durante el desarrollo de los casos de uso sobre el ecosistema IPFS, fue necesario tomar decisiones clave respecto al manejo de conexiones entre nodos, especialmente al tratar con aplicaciones accesibles desde navegadores web. Este anexo describe cómo se llegó a la elección final del protocolo de transporte utilizado en la arquitectura para permitir conexiones desde un nodo web (en navegador) hacia un nodo independiente.

Al comienzo del proyecto, la implementación JavaScript de LibP2P ofrecía dos alternativas principales para habilitar esta conectividad.

\paragraph{WebSockets}

La primera alternativa fue el uso de \textbf{WebSockets} \cite{websocket}, un protocolo que permite establecer una conexión persistente y bidireccional a través de una conexión HTTP inicial. No obstante, para que funcione en navegadores modernos, requiere el uso de certificados TLS válidos, lo cual implica contar con un dominio propio y un certificado emitido por una autoridad certificadora.

Dado que la infraestructura buscaba permitir que cualquier usuario pudiera levantar un nodo de forma autónoma, sin depender de servidores centralizados o dominios externos, esta alternativa fue descartada.

\paragraph{Hole Punching (Relay)}

La segunda opción fue el uso de la técnica conocida como \textbf{Hole Punching} \cite{hole-punching}, mediante el uso de nodos \textit{Relay}. Estos permiten que nodos detrás de NAT o firewalls (como los navegadores) puedan conectarse a otros nodos a través de un intermediario. 

Aunque este enfoque era prometedor, su implementación en JavaScript aún era inestable al momento del desarrollo del proyecto, lo que imposibilitó su adopción efectiva.

\paragraph{AutoTLS}

Frente a estas limitaciones, surgió durante el desarrollo el soporte experimental para una nueva alternativa: \textbf{AutoTLS} \cite{autotls}. Este mecanismo permite obtener automáticamente un certificado TLS asociado al ID del nodo, utilizando el protocolo ACME \cite{acme}. LibP2P provee además un servicio DNS público que resuelve los dominios requeridos para todos los nodos participantes, lo que permite establecer conexiones WebSocket seguras sin intervención del usuario.

Esta solución fue adoptada inicialmente y permitió avanzar con el desarrollo y las pruebas. No obstante, AutoTLS presentó dos inconvenientes críticos. Por un lado, su funcionamiento resultó inestable, ya que se trata de una funcionalidad aún experimental. Por otro, depende de un servicio centralizado para la emisión de certificados, lo cual contradice el principio de descentralización del proyecto. En caso de que dicho servicio quedara fuera de línea, los nodos no podrían conectarse entre sí.

\paragraph{WebRTC-Direct}

Finalmente, se incorporó soporte para una nueva alternativa que ya estaba disponible en otras implementaciones de LibP2P: \textbf{WebRTC}, y en particular su variante \textbf{WebRTC-Direct}, diseñada para conectar directamente navegadores con nodos independientes.

WebRTC es un conjunto de estándares abiertos y APIs que permiten a los navegadores establecer conexiones seguras \textit{peer-to-peer} sin necesidad de certificados TLS. LibP2P adapta este protocolo para permitir exactamente lo que se requería: conexiones directas desde el navegador sin intermediarios ni servidores externos.

Hasta ese momento, incluso los propios desarrolladores de LibP2P recomendaban utilizar implementaciones en otros lenguajes, como Go, ya que la versión JavaScript no ofrecía aún una solución madura para este tipo de conectividad \cite{differences-nodejs-browser}. Sin embargo, cambiar de lenguaje no era una opción viable, ya que el ecosistema del proyecto —incluyendo OrbitDB— está fuertemente vinculado al entorno JavaScript, y además se requería compatibilidad con entornos web.

Con la integración de WebRTC-Direct fue posible reemplazar completamente a AutoTLS y establecer conexiones estables y descentralizadas desde navegadores hacia nodos independientes, cumpliendo tanto con los requisitos técnicos como con los principios filosóficos del proyecto.
