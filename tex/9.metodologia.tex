\section{Metodología}

El desarrollo se dividió en sprints semanales para los cuales utilizamos un tablero Kanban en Github Projects donde fuimos agregando las tareas a realizar para cada caso de uso. Se realizaron reuniones semanales fijas que se usaron como punto de control, donde se revisó lo hecho durante la semana y definimos pasos a seguir para las siguientes. También nos fue útil para detectar posibles ajustes o cambios de rumbo que fueron surgiendo a lo largo del trabajo. Para estas reuniones realizamos minutas que nos sirvieron para resumir lo tratado en cada una y tener un claro los pasos a seguir después de las mismas.

Durante el descubrimiento de las funcionalidades de cada caso de uso realizamos \textit{User Story Mappings} para organizar las tareas de realizar.

La modalidad fue en su mayoría virtual y asincrónica (excepto por la reunión semanal antes
mencionada en la cual los integrantes del trabajo nos reunimos sincrónicamente). Nos mantuvimos en constante comunicación a través de un servidor de Discord y, también se realizaron sesiones de \textit{pair} y \textit{mob-programming} en distintas ocaciones.