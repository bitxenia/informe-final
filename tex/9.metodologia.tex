\section{Metodología aplicada}

La metodología aplicada para la gestión del proyecto fue una versión aproximada a Scrum. El desarrollo se dividió en sprints semanales para los cuales utilizamos un tablero Kanban en Github Projects donde se fueron agregando las tareas a realizar para cada caso de uso y ecosistema. Se realizaron reuniones semanales fijas que se usaron como punto de control, donde se revisó lo hecho durante la semana y definimos pasos a seguir para las siguientes. También nos fue útil para detectar posibles ajustes o cambios de rumbo que fueron surgiendo a lo largo del trabajo. Al finalizar cada una de estas reuniones realizamos minutas que nos sirvieron para resumir lo tratado en cada una y tener en claro los pasos a seguir después de las mismas.

Durante el descubrimiento de las funcionalidades de cada caso de uso realizamos \textit{User Story Mappings} (USM) para organizar las tareas de realizar.

Los distintos artefactos que fueron surgiendo durante el desarrollo del trabajo fueron almacenados en un Google Drive compartido entre todo el equipo. Dentro del mismo se pueden encontrar: minutas de reuniones, cronogramas, USM, entre otros.

Para el control de versiones del código se creó una organización en Github en la que se fueron creando repositorios para los distintos paquetes que integraron el trabajo.

La modalidad fue virtual y asincrónica. Nos mantuvimos en constante comunicación a través de un servidor de Discord y también se realizaron sesiones de \textit{pair} y \textit{mob-programming} en distintas ocasiones.