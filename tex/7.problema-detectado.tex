\section{Problema detectado y/o faltante}

Las soluciones de infraestructura actuales, como AWS, Azure o Google Cloud, presentan barreras significativas para su adopción por parte de proyectos con recursos limitados, como iniciativas independientes, educativas o comunitarias. Estas barreras están relacionadas principalmente con los costos operativos, la dependencia de infraestructura centralizada y la vulnerabilidad frente a fallas o interrupciones del servicio.

\subsection{Costo y sostenibilidad}

La mayoría de las aplicaciones con requerimientos de alta disponibilidad dependen de servicios centralizados en la nube, los cuales implican costos mensuales elevados incluso en etapas tempranas del desarrollo. Este modelo económico desalienta la creación y mantenimiento de servicios comunitarios o de bajo presupuesto, restringiendo su escalabilidad o continuidad en el tiempo.

A su vez, la dependencia de servidores centralizados genera un punto único de mantenimiento y financiamiento que puede convertirse en un cuello de botella. En escenarios donde no existe un respaldo institucional o comercial fuerte, la sostenibilidad del servicio queda comprometida.

\subsection{Interrupciones e infraestructura crítica}

La arquitectura tradicional basada en cliente-servidor implica una fuerte dependencia de la disponibilidad continua de uno o varios nodos centrales. Estas arquitecturas son susceptibles a interrupciones por mantenimiento, errores de configuración, fallas de \textit{hardware} o problemas de conectividad. En muchos casos, un único incidente puede volver una aplicación completamente inaccesible para todos sus usuarios.

Esto evidencia la necesidad de diseñar infraestructuras más resistentes a fallas, donde la continuidad operativa no dependa de un único punto de control.

\subsection{Centralización y control de acceso}

La centralización también facilita el control externo sobre los servicios. Aplicaciones y plataformas pueden ser bloqueadas o restringidas en determinados contextos geográficos o políticos simplemente mediante la identificación y bloqueo de sus puntos de acceso. Esto representa una amenaza para la disponibilidad global y el acceso libre a herramientas comunitarias.

Un ejemplo representativo es el de Wikipedia, cuyo acceso ha sido bloqueado —de forma total o parcial— en diversos países, debido a restricciones impuestas sobre determinados contenidos \parencite{censorship-wikipedia}.

\subsection{Accesibilidad tecnológica}

Finalmente, muchas de las tecnologías necesarias para implementar infraestructuras distribuidas aún requieren conocimientos técnicos avanzados para su configuración, despliegue y operación. Esta complejidad técnica representa una barrera de entrada tanto para usuarios como para desarrolladores que podrían beneficiarse de este tipo de arquitectura, pero que no cuentan con los recursos o conocimientos necesarios para adoptarla.