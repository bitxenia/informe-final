\section{Problema detectado y/o faltante}
Los servicios actuales que proveen de infraestructura tienden a ser muy costosos para las pequeñas comunidades que necesitan tener un servicio con alta disponibilidad, accesible para todos y barato de escalar. Actualmente tampoco existe un estándar para aplicaciones cuyo stack sea completamente distribuido usando peer-to-peer.

\subsection{Costos}
La inversión para el mantenimiento de servidores puede ser un obstáculo a la hora de proveer una red a sus usuarios cuando se trata de una aplicación pequeña o startup. En estos casos, es común sacrificar la escalabilidad en pos de mantener los costos del alojamiento de la aplicación bajos.

En el otro extremo del espectro, se encuentran las aplicaciones en declive. Debido a la falta de incentivos financieros para justificar el mantenimiento, muchas veces la solución es desconectar los servidores definitivamente. Esto es especialmente frecuente en videojuegos multijugador que no cuentan con la capacidad de crear servidores dedicados. En tales situaciones, la empresa propietaria puede desactivar los servidores, lo que resulta en que el videojuego se vuelva parcialmente o completamente inutilizable \cite{dead-games}.

\subsection{Interrupciones del servicio}
Dada la naturaleza del modelo cliente-servidor, no es infrecuente que estas redes se vuelvan inaccesibles. Esto puede ocurrir deliberadamente por temas de mantenimiento, o accidentalmente debido a modificaciones en la aplicación del servidor durante el mantenimiento o actualización de la misma.

Uno de los episodios recientes de mayor revuelo ocurrió el 4 de Octubre de 2021, día en el que la familia de aplicaciones de Meta -Whatsapp, Facebook e Instagram -, estuvo fuera de servicio por seis horas. Esto ocasionó que mas de 3500 millones de usuarios se vieran afectados. En un articulo posterior, Meta reveló que la causa se debió a una falla en la herramienta de auditoría de los comandos que se envían a la red global de datacenters de Meta, la cual evitó que se pudiera detener el comando que desconectaba los servidores de la red. \parencite{meta-offline}

\subsection{Zonas de censura}
Es común que aplicaciones o sitios web comunitarios sean sometidos a su censura. La centralización de los servicios ayuda a que sea mucho más fácil bloquear su acceso, dado que es más fácil identificar y bloquear un único punto de acceso. En contraste, los sistemas descentralizados, suelen ser más resistentes porque no dependen de un servidor o servicio central que pueda ser intervenido fácilmente.

Uno de los casos más conocidos y vigentes es la censura de Wikipedia. Donde algunos gobiernos bloquean su acceso en un determinado lenguaje y otros en su totalidad. \parencite{censorship-wikipedia}
