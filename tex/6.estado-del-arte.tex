\section{Estado del Arte}
En esta sección describiremos en qué se diferencian las aplicaciones descentralizadas de aquellas centralizadas, cuáles son las ventajas (y desventajas) del modelo de aplicación distribuido, y qué tecnologías existen actualmente para asistir en la creación de dichas aplicaciones. 

\subsection{Introducción a arquitecturas de red}
La comunicación entre distintas computadoras requiere una coordinación efectiva entre todas las partes involucradas, el uso de protocolos que estandaricen la forma en que se transmite la información, y una infraestructura que permita enviar cada bit desde el origen hasta su destino. Para ofrecer servicios a través de Internet, es necesario diseñar una red bien estructurada que cumpla con estos requisitos. En este contexto, existen dos arquitecturas principales que permiten alcanzar dicho objetivo.

\subsubsection*{Cliente-Servidor}

El modelo \textit{Cliente-Servidor} es ampliamente utilizado en la mayoría de las aplicaciones disponibles en Internet. Consiste en un nodo central, el \textbf{servidor}, encargado de gestionar la interacción tanto entre los usuarios como entre cada usuario y el propio servidor. En este esquema, los demás nodos actúan como \textbf{clientes}, solicitando servicios o recursos que el servidor proporciona.

Este modelo se clasifica como una arquitectura centralizada, ya que la subred depende directamente del servidor. En caso de que este nodo central falle, los clientes pierden la capacidad de comunicarse entre sí o acceder a los servicios, lo que representa un punto único de fallo.

Entre los servicios más comunes que adoptan esta arquitectura se encuentran la World Wide Web (HTTP/HTTPS), el correo electrónico (SMTP, IMAP) y el sistema de DNS, entre otros.

\subsubsection*{Peer-to-Peer}
El modelo \textbf{peer-to-peer(P2P)} consiste en una red \textbf{descentralizada} que tiene distintos nodos (pares) capaces de comunicarse sin necesidad de un nodo central, por lo que se puede considerar que cada nodo cumple la función tanto de servidor como de cliente.

\paragraph{BitTorrent} El servicio más utilizado que implementa este modelo es la red de BitTorrent, que implementa el protocolo del mismo nombre para compartir archivos entre pares. Esta red logra que el mismo nodo que descarga un contenido de la red sea a la vez el servidor para otro nodo que quiera acceder a ese contenido. 

\subsection{Diferencias y ventajas de cada arquitectura}
Ambos modelos tienen ventajas y desventajas, y por lo tanto distintos casos de uso. El modelo cliente-servidor actualmente es la arquitectura mas utilizada.

\paragraph{Resiliencia}
Cuando un servidor se encuentra fuera de servicio, toda la red que depende de él no funcionará en tanto no se restaure el servidor. Para contrarrestar esta vulnerabilidad del modelo, se desarrollaron métodos a lo largo de los años. Una manera de evitar que la red se vuelva inoperativa es la de alojar instancias del servidor en diferentes zonas geográficas. Además, se pueden implementar medidas para evitar la caída del servidor, como las técnicas de balanceo de cargas y fuentes de energía alternativas para evitar eventuales cortes de electricidad.

No obstante, una red peer-to-peer puede ser incluso más robusta. Si hay suficientes pares en la misma y están lo suficientemente dispersos geográficamente, la desconexión de uno de ellos no desactiva toda la red. Esto permite que el modelo peer-to-peer pueda ser resistente a cortes de energía masivos y desastres naturales, lo que lo hace un modelo ideal para servicios prioritarios.

Cabe destacar que en una red de una cantidad limitada de pares, en donde no hay redundancia del contenido que se distribuye, es posible que al desconectar uno de los pares parte del contenido se vuelva no disponible. Por lo tanto, si bien la red seguirá activa, no tendrá toda la funcionalidad que si puede ofrecer un servidor mientras siga en línea.

\paragraph{Escalabilidad}
La escalabilidad de una red peer-to-peer aumenta con la cantidad de nodos disponibles, dado que hay más recursos y, en una red bien diseñada, la carga se distribuye equitativamente. En un modelo cliente-servidor, garantizar escalabilidad se puede tornar costoso. Un servidor con mayor capacidad para comunicaciones entrantes y volumen de información tiene hardware de un costo mayor. En casos de aplicaciones de uso masivo puede ser necesario multiplicar la cantidad de nodos servidores para satisfacer la demanda de clientes.

\paragraph{Control del contenido}
Un servidor, al ser la pieza central de la red a la cuál pertenece, debe soportar múltiples conexiones en simultáneo. Para aplicaciones de alto tráfico, esto requiere una infraestructura que los usuarios suelen no poseer. Una solución es tercerizar el alojamiento de la aplicación servidor en plataformas de Cloud Hosting como pueden ser AWS, Azure, Google Cloud, entre otras. Estos servicios mantienen los servidores de numerosas aplicaciones de Internet, y por lo tanto, tienen la capacidad de modificar, censurar, o remover cualquiera de ellas si así lo desean.

Una red P2P no sufre de estos problemas, ya que por naturaleza los usuarios son quienes la alojan. Por lo tanto, remover contenido de ella resulta mucho mas complejo. Esto evita la censura en zonas donde el acceso a Internet es controlado y/o limitado, pero también puede incentivar a la distribución de contenido ilegal.

\paragraph{Seguridad}
La seguridad en las aplicaciones cliente-servidor se ha investigado por mucho más tiempo debido a la popularidad de este modelo. Además, al ser centralizado, el propietario del servidor puede bloquear conexiones y eliminar contenido malicioso de su plataforma de forma transparente para los usuarios.

El modelo descentralizado, en cambio, no cuenta con el desarrollo en términos de seguridad. En este caso, el cliente es el responsable de conectarse a redes de confianza, o de hacerlo mediante VPNs (Virtual Private Networks) para mantener el anonimato mientras se integre una red P2P. Sin embargo, cualquiera sea el modelo utilizado por una aplicación, la mayor parte de la seguridad dependerá de que tan segura sea la aplicación.

\paragraph{Persistencia}
Como varias otras propiedades del modelo cliente-servidor, depende de la integridad del servidor. Si el almacenamiento físico de este se ve afectado, los datos pueden perderse definitivamente. Por esta razón, es común tener un respaldo de los datos de la aplicación en otro disco u otro nodo para evitar la pérdida total de datos.

En una red descentralizada, la persistencia depende de la aplicación utilizada. En el caso de BitTorrent, cada nodo que se conecte y descargue un archivo, podrá compartir ese archivo con otros nodos, y por lo tanto ese archivo contará con una redundancia adicional, la cuál crece a medida que más personas descargan ese archivo. A pesar de esto, en casos donde el archivo es poco compartido, puede volverse inaccesible si los nodos que lo contienen se desconectan de la red.

\paragraph{Latencia}
Dada una conexión a Internet promedio, las velocidades manejadas por las aplicaciones cliente-servidor suelen ser aceptables. Sin embargo, en zonas en donde la conexión es escasa, o en casos en donde el servidor está lejos del cliente, la velocidad de transferencia de la aplicación puede verse afectada. Además, no es infrecuente encontrar cortes en videollamadas, juegos, y demás aplicaciones de tiempo real que siguen esta arquitectura. Como en la mayoría de defectos del modelo cliente-servidor, se puede solucionar agregando múltiples instancias del servidor. Por ejemplo, es común almacenar  películas, videos y demás contenido de aplicaciones de streaming en distintos servidores de CDN (Content Delivery Network). Estas redes minimizan la distancia entre el usuario y el servidor, agilizando así la transferencia del contenido.

A pesar de los avances en la optimización del modelo cliente-servidor, las redes descentralizadas, cuando son eficientes y están bien pobladas, suelen ofrecer incluso mejores resultados. Esto se debe a que la fuente de un contenido puede estar presente en múltiples nodos, lo que aumenta la probabilidad de que un nodo cercano tenga el contenido solicitado. La velocidad de transferencia que puede proporcionar un vecino con el contenido que requerimos generalmente superará la ofrecida por un servidor.

\paragraph{Costos}
Los costos de alojar una aplicación peer-to-peer suele ser nulo, ya que los mismos usuarios de ella son los encargados de proporcionar la infraestructura de la red.

Al contrario, alojar la aplicación en un servidor conlleva tener un servidor disponible, o bien contratar un servicio de web hosting, cuya tarifa suele aumentar a medida que la aplicación escala. En la mayoría de los casos, el costo termina siendo significativo, por lo que una aplicación descentralizada es una opción viable en escenarios en los que no se desee invertir mucho dinero.

\subsection{Soluciones existentes}

% Mover esta sección para después de ambientes y herramientas?

\subsubsection{IPFS Deploy Action}
Para desplegar un sitio web en IPFS, se puede utilizar un \textit{GitHub Action} hecho por IPFS y lanzado en Febrero de 2025. Dicha herramienta es un script que se ejecuta con cada commit en un repositorio de GitHub, y permite compilar el sitio web, y luego alojarlo en IPFS utilizando un servicio de \textit{pinning}, Filecoin, o bien un clúster de IPFS. Estas opciones se verán en detalle cuando se abarque la arquitectura de despliegue implementada.

Sin embargo cuenta con una serie de desventajas al utilizar clústeres. Por un lado, utilizando la opción de clúster sólo puede instruir a un único nodo para que luego este sincronice el contenido al resto de los nodos. Un conjunto de nodos que colaboran con el despliegue y mantenimiento de un sitio web sólo podrán actualizarse si el nodo central está activo, lo que resulta en un único punto de falla. Por otro lado, utilizar un \textit{GitHub Action} nos ata necesariamente a \textit{GitHub}, lo cuál supone otra limitación para quienes deseen optar por otra alternativa para su repositorio Git.

\subsubsection{Distributed Wikipedia Mirror}
%TODO: redactar
Como sucede con un caso de uso muy similar al repositorio de conocimiento que queremos implementar, el proyecto de \textbf{Distributed Wikipedia Mirror}\cite{distributed-wikipedia-mirror}, el cual consistió en poner una versión de wikipedia en IPFS, únicamente funciona como versión Read-Only, y con snapshots manuales, lo cual es totalmente posible de hacer con la infraestructura que explicamos anteriormente.

\subsection{Ambientes y herramientas}
Existen varios ecosistemas que apuntan a proveer un marco con el cuál desarrollar una aplicación descentralizada. A su vez, cada uno de ellos cuenta con herramientas especializadas para los diferentes tipos de aplicaciones.

\subsubsection{IPFS}

\textbf{IPFS} (InterPlanetary File System) es un conjunto modular de protocolos diseñado para la organización y transferencia de datos en una red peer-to-peer, basado en el principio de \textit{content addressing}, es decir, la recuperación de archivos en función de su contenido y no de su ubicación o identificador arbitrario \cite{ipfs}. Su principal propósito es facilitar la publicación de datos como archivos, directorios y sitios web de forma descentralizada.

Este enfoque representa una alternativa al modelo tradicional de la web, que se basa en el direccionamiento por ubicación (\textit{location-based addressing}), como ocurre con HTTP. Dicho modelo impone limitaciones estructurales que son contrarias a los principios de descentralización, resiliencia y autonomía que caracterizan a las aplicaciones comunitarias y distribuidas.

IPFS, en cambio, propone una red abierta, participativa y sin control centralizado, donde cualquier usuario puede contribuir y operar como nodo. Esta descentralización no solo elimina la dependencia de servicios de terceros para el despliegue de aplicaciones —lo cual puede ser costoso—, sino que también permite a los propios usuarios colaborar activamente con recursos de sus dispositivos para sostener el funcionamiento de la red.

Además, gracias al direccionamiento por contenido, cada archivo en IPFS cuenta con un identificador único (CID), lo que permite un acceso persistente y distribuido desde múltiples ubicaciones. Esta característica mejora la disponibilidad del contenido y fortalece la resistencia frente a intentos de censura o interrupciones del servicio.

En este contexto, IPFS constituye una base tecnológica especialmente adecuada para el desarrollo de aplicaciones descentralizadas. A continuación, se presentan algunas de las herramientas más relevantes que conforman su ecosistema.

\paragraph{libp2p} Colección de protocolos y utilidades para facilitar la implementación de una red peer-to-peer \cite{libp2p}. Entre sus herramientas, se encuentran diferentes mecanismos de seguridad, de transporte, y para descubrimiento de pares. Se creó con IPFS en mente, pero luego se expandió a un conjunto de protocolos independiente, el cual es utilizado por Ethereum actualmente. Los protocolos de interés para este proyecto son:
\subparagraph{Protocolos de transporte} Son los encargados de la comunicación entre nodos, de manera similar a la capa de transporte presente en toda red convencional. Se basan en tipos de transporte ya existentes, adaptados al uso peer-to-peer. Los protocolos principales son TCP, WebSockets y WebRTCDirect.
\subparagraph{Protocolos de descubrimiento de peers} Para encontrar un contenido en IPFS, se necesita saber la dirección del nodo que tiene dicho contenido. El principal protocolo para lograr esto se denomina Distributed Hash Table (DHT) \cite{dht} \cite{kadmelia}. Es un registro clave-valor distribuido en todos los nodos que soporten este protocolo, que contiene la información necesaria para encontrar el contenido deseado. Cada nodo tiene una parte de esta tabla, y deberá preguntar a otros nodos hasta conseguir la dirección del nodo asociada a la dirección del contenido buscado.

\paragraph{Kubo} La implementación principal de IPFS es Kubo \cite{kubo}, una solución hecha en Go. Tiene su propio comando en la terminal, llamado \texttt{ipfs}, y también es utilizado en los demás front-ends de IPFS como IPFS desktop, la aplicación de escritorio de IPFS. Es la más madura y desarrollada de las dos implementaciones de IPFS, y cuenta con más funcionalidades.

\paragraph{IPFS Cluster} Herramienta para orquestar distintos nodos de IPFS, con el objetivo de mantener disponible un contenido en la red de IPFS, aumentando su disponibilidad. Por defecto, cualquier nodo puede modificar la lista de archivos que mantiene el \textit{cluster}.
\subparagraph{Clusters colaborativos} Son clusters que permiten que usuarios puedan colaborar con su nodo y aumentar la disponibilidad del contenido, sin tener permiso para modificar los archivos que se están manteniendo \cite{collaborative-clusters}. Esto es ideal para una aplicación comunitaria debido a esta medida de seguridad que permite al cluster ser apoyado por cualquier usuario de una comunidad.

\paragraph{OrbitDB}
OrbitDB es una base de datos descentralizada y peer-to-peer construida sobre \textbf{IPFS} para el almacenamiento de datos, y utiliza \textbf{libp2p} para la sincronización entre nodos \cite{orbitdb}. Su modelo es de consistencia eventual, lo que significa que los cambios realizados en una base de datos no se reflejan de manera inmediata en todos los nodos de la red, sino que se propagan de forma progresiva con el tiempo.
OrbitDB ha sido desarrollada principalmente dentro del ecosistema de \textit{JavaScript}, lo que facilita su integración en aplicaciones web. Esta característica permite ejecutar nodos directamente en el navegador.

\paragraph{Helia} Para crear una instancia de un nodo de OrbitDB se debe pasar como parámetro una instancia de Helia. Helia \cite{helia} es la implementación de IPFS para Javascript, y, como OrbitDB, también puede ejecutarse en un navegador. A diferencia de Kubo, Helia tiene un enfoque mucho más abierto, permitiendo la configuración del nodo libp2p utilizado internamente de manera abierta. Esto implica un trade-off claro: más configuración permite un nodo de IPFS que se puede adaptar a diferentes escenarios, pero también implica una mayor complejidad a la hora de lograr un funcionamiento correcto del nodo.

\subsubsection{Blockchain}
Tecnología basada en una cadena de bloques de operaciones descentralizada y pública.  Esta tecnología genera una base de datos compartida a la que tienen acceso sus participantes, los cuáles pueden rastrear cada transacción que hayan realizado.\parencite{blockchain}


% \footnote{https://www.iebschool.com/blog/blockchain-cadena-bloques-revoluciona-sector-financiero-finanzas/}


\subsubsection{Alternativas}
\paragraph{Freenet/Hyphanet}
Programa de código abierto para compartir datos peer-to-peer con enfoque en la protección de la privacidad. Opera en una red descentralizada, promocionando la libertad de expresión facilitando la anonimidad de los datos compartidos y eludiendo la censura.\cite{freenet}\cite{hyphanet}
