\section{Introducción}

Hoy en día, al querer desplegar una aplicación o sitio web comunitario, lo más común es hacerlo a través de un servicio de alojamiento (AWS, Azure, Google Cloud, entre otros) por la comodidad y facilidad que estas ofrecen, alquilando sus servidores para guardar y procesar datos.

Esto puede llegar a traer problemas para este tipo de aplicaciones. Uno de estos problemas puede ser monetario, ya que muchas veces estas aplicaciones dependen de donaciones o voluntarios para sustentarse, como es el caso de Wikipedia. Como también puede suceder que se encuentre en una zona de censura, lo cual facilita su bloqueo al ser servicios centralizados; entre otros problemas más.

Sin embargo, existen otros ecosistemas alternativos que se asemejan mucho más a la filosofía de estas aplicaciones, y que ayudan a combatir estos problemas. En donde las aplicaciones pueden estar alojadas por sus propios usuarios, donando su computo o espacio, y así logrando una descentralización.

En el siguiente documento presentamos un análisis sobre la infraestructura existente, donde es posible la implementación de plataformas para el despliegue de este tipo de aplicaciones, recabando las bondades y desventajas que cada una tiene. 

Para esto se crearon diferentes casos de uso que representan posibles aplicaciones sobre esta metodología alternativa analizando su viabilidad. Entre ellos, se encuentran un sitio web estático, una enciclopedia colaborativa y una aplicación de comunicación en tiempo real.