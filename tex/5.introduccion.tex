\section{Introducción}

El presente trabajo tiene como objetivo explorar alternativas de infraestructura para el desarrollo y despliegue de aplicaciones comunitarias, descentralizadas y distribuidas. A diferencia de los servicios centralizados convencionales, este tipo de aplicaciones enfrentan desafíos particulares que no siempre pueden resolverse de manera eficaz mediante las plataformas tradicionales.

En la actualidad, la mayoría de los servicios web se apoyan en proveedores de infraestructura como AWS, Azure o Google Cloud, que ofrecen soluciones ágiles para el desarrollo, despliegue y mantenimiento de aplicaciones. Sin embargo, este modelo no siempre resulta adecuado para aplicaciones comunitarias, entendidas como aquellas creadas, mantenidas y/o gobernadas por una comunidad. En estos casos, la toma de decisiones está distribuida entre los participantes, orientada al bien común y basada en la colaboración entre usuarios. Esta estructura permite que la evolución de la aplicación responda directamente a las necesidades de su comunidad.

Las aplicaciones comunitarias suelen operar con presupuestos limitados, sustentándose a través de voluntariado o donaciones, y pueden verse afectadas por políticas de censura u otras restricciones en determinadas regiones. Además, su propósito suele estar alineado con valores como la apertura y el acceso equitativo a la información, lo que entra en tensión con las lógicas de control y dependencia que caracterizan a las infraestructuras centralizadas.

En este contexto, surgen ecosistemas alternativos —basados en tecnologías \textit{peer-to-peer}— que permiten a las aplicaciones ser alojadas y mantenidas directamente por sus usuarios, compartiendo recursos como almacenamiento y conectividad. Este enfoque distribuye la infraestructura entre múltiples nodos, eliminando puntos únicos de fallo, aumentando la resistencia a la censura y reduciendo los costos operativos.

Con el objetivo de evaluar estas tecnologías, se diseñaron e implementaron diversos casos de uso representativos, como un sitio web estático, un repositorio de conocimiento y un mensajero en tiempo real. Estos casos permitieron analizar la viabilidad técnica y operativa de construir aplicaciones comunitarias completamente descentralizadas, sin depender de infraestructura de terceros.

Este documento presenta un análisis detallado de la infraestructura disponible para la implementación y el despliegue de aplicaciones comunitarias descentralizadas. A través de casos de uso concretos, se examinan las capacidades y restricciones de cada ecosistema evaluado, destacando su aplicabilidad y potencial que ofrece este enfoque alternativo.