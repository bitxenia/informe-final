\section{Trabajos futuros}

En base al análisis realizado, se detectaron diversas lineas de trabajo futuras de las cuales se puede continuar el desarrollo.

\subsection{Mejoras a los casos de uso}

A continuación se proponen mejoras para los casos de uso implementados, los cuales, por temas de tiempo, no se llegaron a implementar.

\subsubsection{AstraDB}
%TODO: redactar

Sin embargo este enfoque presenta algunas limitaciones o aspectos a mejorar que son importantes destacar.

Algo a notar es la falta de una conexión entre nodos web, lo cual sería algo muy útil, ya que los mismos usuarios web utilizando la aplicación en un determinado momento, podrian ayudar a proveer de las bases de datos a nuevos usuarios que se conecten y no dejar que solo los nodos independientes puedan hacerlo, lo cual seguría muy bien la filosofía de IPFS, sin embargo esto requeriría que los nodos web puedan escuchar conexiones entrantes y esto es únicamente posible haciendo uso de un \textbf{Relay} como explicamos anteriormente y que actualmente no fue posible obtener un funcionamiento estable en el contexto de nuestro proyecto. Igualmente sería algo muy fácil de implementar una vez que su funcionamiento sea estable y agregaría esta característica que mejoraría la infraestructura.

Para la solución implementada, la seguridad no fue una prioridad. En consecuencia se presentan algunos problemas que están relacionados con esto. La solución da total libertad a la creación y agregado de información a las bases de datos sin ninguna limitación, junto a que los colaboradores deben almacenar la totalidad de la base de datos, por lo que se explicó anteriormente de la falta de \textit{Proof of storage}, lo cual de por si es un limitación. En consecuencia es muy fácil realizar un ataque de spam, usando el espacio almacenado de los colaboradores, al crear nuevas \textit{Keys} o agregar contenido sin importancia a cada una de ellas. Esto podría mitigarse si se utilizara un consenso a la hora de aceptar un cambio o no a la base de datos, dado que si el cambio no se acepta, entonces no se distribuye y no se efectúa, sin embargo, como se explicó, esto requiere un foco de seguridad y para mostrar su uso en el ecosistema no fue el punto principal.

Relacionado con lo anterior, permitir la modificación libre de toda la información puede que no sea lo deseado para otro tipo de aplicaciones. Si se quisiera crear un blog personal o de microblogging, se desearia que solo  pueda moficar la información el dueño del blog y que los colaboradores solo ayuden con la persistencia del mismo. Esto puede ser una posible mejora y es posible si se puediera optar que las bases de datos de las keys tengan el permiso de moficiacion unicamente al usuario que las creó, como tambien referenciar a la \textit{Address} de la base de datos creada, ya que en este caso si depede de su creador, por lo que se explicó en anterioridad en la representación de los datos.

Tambien tiene una limitacion con que no se pueden crear cosas privadas, dado que el almacenamiento esta en la red de ipfs y cualquiera puede acceder a el y la falta de encirptacion, hace que para casos de uso donde se requiera privacidad de informacion no es lo ideal.

\subsubsection{Astrawiki-eth}

Teniendo en cuenta que es el caso de uso donde mayor volumen de datos se transfieren, por lo tanto mayor costo de gas, una mejora involucra la disminución de este costo. Se podría disminuir haciendo uso de los clones de \textit{smart contract}s según lo propuesto en el ERC-1167 \cite{erc-1167}. En resumen, lo que se propone es que haya un contrato como ''plantilla'' al cual se lo copia cada vez que, por ejemplo, se crea un artículo nuevo.

\subsubsection{Astrachat-eth}

Una de las principales mejoras que se pueden implementar sobre este caso de uso es respecto a la experiencia de uso para frontends en navegadores web. El hecho que para cada mensaje enviado sea necesario confirmar una transacción por medio de la wallet hace que la aplicación pierda el factor de \textit{real-time}. Una posible solución a este problema es que se pueda crear un balance en un fondo común (con dinero de los usuarios) de forma que cada vez que se envíe un mensaje se tome de este fondo para pagar la transacción. Esta mejora haría más fácil la integración con otros frontends que no corren en navegadores web como \textit{Astrachat-cli} aunque también queda investigar cómo conectar una wallet a los mismos.

Por otro lado, en base al costo por transacción (visto en la sección \ref{performance-blockchain}) se podría utilizar alguna red cuyos precios de gas por transacción sean menores. Una de estas redes --basada en Ethereum-- es la \textit{Ronin Network} \cite{ronin-network} \cite{ronin-network-whitepaper} que a principio de este año (2025) abrió su red para que cualquier usuario pueda implementar aplicaciones en la misma.

\subsection{Mejora para clusters colaborativos}
%TODO: Redactar
Como se mencionó previamente en el análisis de IPFS, los clusters colaborativos son una alternativa para lograr la persistencia y disponibilidad de archivos, a través del "pinneo" de nodos colaborativos. La cual va muy de la mano con la filosofía de las aplicaciones comunitarias, ya que potencialmente la propia gente que utiliza la aplicación sea la que la mantenga colaborativamente. El problema es que actualmente casi no se utiliza esta alternativa, lo cual lleva a que por ejemplo, no haya una manera fácil de utilizarla, sin embargo esto tiene sus razones.

\paragraph{IPFS Cluster}

Para entender mejor por qué no se utiliza mucho esta forma de "pineo" y cuales son sus limitaciones, primero hay que entender que es lo que se utiliza. IPFS cuenta con una aplicación distribuida llamada IPFS Cluster, esta funciona a la par de los nodos de IPFS y permite mantener un conjunto global de "pines", alocando inteligentemente los items a sus nodos.

Gracias a esta aplicación es posible la creación de clusters colaborativos. Los clusters colaborativos permiten que \textit{peers} individuales, que no son de confianza, se unan a un clúster IPFS como "seguidores" (sin permiso para modificar o editar el conjunto de pines).

Para crear este cluster es necesario primero contar con algunos nodos confiables, aquellos que van a tener permitido modificar qué archivos son los que pertenecen al conjunto de "pines". Y es esta nuestra primera limitación, ya que no se puede permitir que cualquier nodo pueda modificar este conjunto, entonces es necesario contar con nodos propios que tengan ese rol.

Otra cosa muy importante que se puede configurar sobre el cluster es el \textit{replication factor}, este ajuste permite indicar a cuantos nodos se va a replicar cada archivo. Esto es muy importante porque permite que no sea obligatorio replicar la totalidad de los archivos en todos los nodos. 

Sin embargo existe una limitación muy importante, IPFS Cluster no verifica si los seguidores realmente están "pineando" o no el contenido, ni cuánto espacio tienen ni otras métricas. Confía en todo lo que dicen. Eso significa que un seguidor malintencionado siempre puede fingir que tiene suficiente espacio, que se le asignen pines o fragmentos, pero al final no "pinea" nada. Un grupo formado por voluntarios aleatorios y que no son de confianza no puede usar, por ejemplo, factor de replicación de tres, porque esos 3 miembros podrían simplemente estar fingiendo "pinear" cosas. Es por eso que el escenario que mejor funciona es dejando que todos "pineen" todo y esperar que algunos de los \textit{peers} no sean maliciosos.

Justamente de esa manera es como funcionan los clusters públicos actualmente, visibles en la página de IPFS \cite{collaborative-clusters}, donde hay algunos que piden gigabytes, y hasta algunos terabytes de información, para poder colaborar. Esto es algo totalmente inviable para que la gente pueda colaborar hasta un cierto almacenamiento, limitando mucho la posibilidad de distribución de colaboradores.

\paragraph{Deployado, seguimiento y descubrimiento}

Habiendo hablado ya de las limitaciones más técnicas que se prensentan al usar cluster colaborativos, hay otro tipo de limitación que es el la falta de facilidad a la hora de deployar, seguir y descubrir aplicaciones colaborativas que usen IPFS cluster.

En primer lugar hablemos del deployado, actualmente si se quiere deployar la aplicación en un cluster colaborativo la única forma que existe es hacerlo de forma manual y crear un servicio que se ocupe de actualizar el conjunto de "pines" cuando sea necesario actualizarse. Si bien es factible, no es algo cómodo para todos y limita a decidirse ir por este camino. Lo ideal sería, que exista un servicio, parecido al ofrecido en los servicios de \textit{pinning}, como es el caso de Fleek \cite{fleek}, donde al actualizar el repositorio ya se deployen los cambios. Facilitando asi su deployado en el cluster colaborativo. Se podría pensar como un servicio que corra sobre los nodos \textit{trusted} donde, cuando un nodo lider detecte que se hizo un cambio, se "buildee" la página web por ejemplo, y se despliegue en el cluster comunitario.

Parecido a lo anterior, alguien que quiera colaborar necesita ejecutar comandos manualmente que para mucha gente no sabe lo que está haciendo, lo cual también termina desincentivando y generando que menos gente decida por colaborar.

Por último no existe una manera fácil de presentar y descubrir nuevos o existentes proyectos comunitarios que la gente pueda optar por colaborar. Actualmente el único lugar donde prensentan proyectos públicos para colaborar es la página de IPFS \cite{collaborative-clusters}, que ademas se ser bastante simple, depende de ellos si agregan un nuevo proyecto para presentar en esa página. Donde lo ideal sería que puedan publicarse nuevos proyectos junto a una descripción y que la gente pueda conocerlos y optar por colaborar de una manera mucho más sencilla.

\paragraph{Conclusión}

Con esto concluimos que se presenta la oportunidad para una linea de trabajo futura focalizada en la mejora para clusters colaborativos. 

En primer lugar una mejora técnica donde se modifique o se construya por encima de IPFS cluster, que permita principalmente que en un clúster comunitario se asegure el "pineo" de información sin necesidad de tener que obligar a todos los colaboradores a "pinear" la totalidad de los archivos.

En segundo lugar, la creación de una plataforma o aplicación, que permita la facilitación del deployado, seguimiento y descubrimiento de aplicaciones colaborativas.

\subsection{Análisis del consumo de energía en la red de IPFS}

Como se mencionó anteriormente, no hay análisis detallado del uso de energía en la red de IPFS. Un posible trabajo involucra calcular una huella de carbono aproximada, el consumo promedio de energía, entre otras métricas. De ello puede desprenderse un análisis comparativo entre diferentes ecosistemas como Blockchain, y cloud hosting tradicional.

\subsection{Blockchain para aplicaciones comunitarias}

En la búsqueda de una blockchain comunitaria nos encontramos con un servicio llamado \textit{Filecoin} \cite{filecoin} el cual provee de incentivos monetarios para aquellos que cedan espacio en sus discos a guardar archivos de otras personas. Es abierto para que cualquier persona pueda unirse a la red peer-to-peer que proveen. Funciona por arriba de IPFS, utiliza tecnologías de Blockchain para los incentivos y dar garantía que los datos están realmente guardados en los nodos. Nuestra propuesta de trabajo futuro es que se cree una tecnología basada en Blockchain que no necesite del incentivo monetario para funcionar y que además permita la ejecución de código.

\subsection{Análisis de Freenet como ecosistema}

Si en un futuro Freenet finalmente sale en una versión estable, creemos que sería un buen ecosistema al cual realizar un análisis como el de este trabajo y hasta más en profundidad con otro tipo de aplicaciones distribuidas y descentralizadas.
