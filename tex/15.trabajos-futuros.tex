\section{Trabajos futuros}

A partir del análisis realizado, se identificaron diversas líneas de trabajo que abren la posibilidad de continuar y ampliar el desarrollo presentado.

\subsection{Mejoras a los casos de uso}

A continuación se proponen mejoras concretas para los casos de uso implementados. Si bien no pudieron ser abordadas en esta instancia por cuestiones de tiempo, representan oportunidades claras para fortalecer y extender la solución actual.

\subsubsection{AstraDB}

El enfoque adoptado para AstraDB presenta ciertas limitaciones que abren la puerta a posibles mejoras futuras. A continuación, se enumeran algunas de las más relevantes:

Una primera mejora posible sería habilitar conexiones entre nodos web. Actualmente, la infraestructura sólo permite que los nodos independientes sirvan las bases de datos al resto de los usuarios. Poder establecer conexiones directas entre navegadores permitiría que los mismos usuarios web que estén utilizando la aplicación en simultáneo puedan colaborar en la distribución de datos, alineándose aún más con la filosofía descentralizada de IPFS. Esta funcionalidad requiere que los nodos web sean capaces de aceptar conexiones entrantes, lo cual solo es posible mediante el uso de un \textbf{Relay}. En el contexto de este proyecto, no se logró un funcionamiento estable del sistema de Relay, pero su incorporación una vez estabilizado sería sencilla e incrementaría significativamente la resiliencia y disponibilidad del sistema.

Por otra parte, la solución actual no priorizó aspectos de seguridad. Toda persona que conozca el nombre de una entidad puede agregar información a su base de datos correspondiente, y los colaboradores deben almacenar su contenido completo. Al no contar con un sistema de \textit{Proof of Storage}, esto abre la posibilidad a ataques de spam: un usuario malicioso podría crear numerosas claves o llenar las existentes con información irrelevante, ocupando el almacenamiento de los colaboradores. Este riesgo podría mitigarse implementando un mecanismo de consenso, mediante el cual los cambios propuestos deban ser aprobados antes de ser distribuidos y aceptados. No obstante, este tipo de enfoque requiere un análisis más profundo centrado en la seguridad, que escapa a los objetivos de esta implementación inicial.

Relacionado con lo anterior, también se identifica como posible mejora el soporte para casos de uso donde se requiera control de edición más estricto. Actualmente, todas las claves son editables por cualquier usuario. Para aplicaciones como blogs personales o microblogs, sería deseable que solo el creador de una clave pueda modificar su contenido, mientras que el resto de los nodos actúan únicamente como replicadores. Esto podría lograrse incorporando controles de acceso más específicos, por ejemplo, estableciendo que la base de datos asociada a cada clave solo acepte modificaciones del nodo que la creó. Dado que la dirección de una base de datos incluye la clave pública de su creador, esta validación es técnicamente posible dentro del modelo de OrbitDB.

Finalmente, la solución actual no contempla la posibilidad de datos privados. Todo el contenido se almacena públicamente en la red IPFS y no se aplica ningún mecanismo de cifrado. Por tanto, no es adecuada para aplicaciones que requieran confidencialidad. La incorporación de esquemas de cifrado a nivel de aplicación podría ser una solución viable en futuros desarrollos, permitiendo que sólo los usuarios autorizados puedan leer ciertos contenidos.

\subsubsection{Astrawiki-eth}

Teniendo en cuenta que es el caso de uso donde mayor volumen de datos se transfieren, por lo tanto mayor costo de gas, una mejora involucra la disminución de este costo. Se podría disminuir haciendo uso de los clones de \textit{smart contract}s según lo propuesto en el ERC-1167 \cite{erc-1167}. En resumen, lo que se propone es que haya un contrato como ''plantilla'' al cual se lo copia cada vez que, por ejemplo, se crea un artículo nuevo.

\subsubsection{Astrachat-eth}

Una de las principales mejoras que se pueden implementar sobre este caso de uso es respecto a la experiencia de uso para frontends en navegadores web. El hecho que para cada mensaje enviado sea necesario confirmar una transacción por medio de la wallet hace que la aplicación pierda el factor de \textit{real-time}. Una posible solución a este problema es que se pueda crear un balance en un fondo común (con dinero de los usuarios) de forma que cada vez que se envíe un mensaje se tome de este fondo para pagar la transacción. Esta mejora haría más fácil la integración con otros frontends que no corren en navegadores web como \textit{Astrachat-cli} aunque también queda investigar cómo conectar una wallet a los mismos.

Por otro lado, en base al costo por transacción (visto en la sección \ref{performance-blockchain}) se podría utilizar alguna red cuyos precios de gas por transacción sean menores. Una de estas redes --basada en Ethereum-- es la \textit{Ronin Network} \cite{ronin-network} \cite{ronin-network-whitepaper} que a principio de este año (2025) abrió su red para que cualquier usuario pueda implementar aplicaciones en la misma.

\subsection{Mejora para clusters colaborativos}

Como se analizó previamente en el contexto de IPFS, los clusters colaborativos representan una alternativa viable para garantizar la persistencia y disponibilidad de archivos mediante el \textit{pinning} distribuido entre nodos participantes, permitiendo así el \textit{hosting} comunitario.

No obstante, esta alternativa aún presenta una adopción limitada en la práctica, debido a diversas restricciones técnicas y operativas, que se detallan a continuación.

\paragraph{IPFS Cluster}

IPFS Cluster permite configurar el \textit{replication factor}, es decir, la cantidad deseada de réplicas para cada archivo dentro del clúster. Este parámetro resulta útil para optimizar el uso del almacenamiento, evitando la duplicación innecesaria de datos en todos los nodos. Sin embargo, presenta una limitación crítica: IPFS Cluster no implementa mecanismos de verificación que aseguren que los nodos seguidores estén efectivamente realizando \textit{pinning} del contenido asignado. Tampoco valida el espacio disponible, la integridad de los datos ni el cumplimiento efectivo del acuerdo de replicación.

Como consecuencia, un nodo malicioso puede simular estar colaborando sin almacenar realmente ningún dato. Esta falta de garantías impide aplicar con seguridad factores de replicación bajos, ya que el contenido podría volverse inaccesible si los nodos asignados no cumplen su rol. Por este motivo, la estrategia más común es replicar todo el contenido en todos los nodos, lo cual restringe severamente la escalabilidad del sistema.

Este modelo es el que se observa actualmente en los clusters públicos listados por el proyecto IPFS \cite{collaborative-clusters}, donde a los colaboradores se les solicita disponer de capacidades de almacenamiento que pueden variar desde varios \textit{gigabytes} hasta \textit{terabytes}, limitando significativamente la participación comunitaria a gran escala.

\paragraph{Despliegue, seguimiento y descubrimiento}

Más allá de las limitaciones técnicas, existen también obstáculos operativos en cuanto a la facilidad de despliegue, monitoreo y descubrimiento de proyectos colaborativos que utilizan IPFS Cluster.

Si bien nuestra herramienta de infraestructura de despliegue facilita la publicación de contenido en clusters colaborativos, la incorporación de nuevos colaboradores continúa requiriendo conocimientos técnicos avanzados y la ejecución manual de comandos, lo que constituye una barrera de entrada para usuarios menos experimentados. El desarrollo de herramientas más accesibles y orientadas a la experiencia del usuario permitiría ampliar significativamente la red de participantes.

Asimismo, no existe actualmente una plataforma abierta y descentralizada que permita registrar y descubrir fácilmente proyectos comunitarios. Los únicos clusters públicos disponibles se encuentran listados en la página oficial de IPFS \cite{collaborative-clusters}, donde la inclusión de nuevos proyectos depende de una gestión centralizada. Una solución que permita registrar proyectos de forma abierta, incluyendo descripciones, requisitos de colaboración y enlaces de participación, facilitaría enormemente la adopción y el involucramiento comunitario en una variedad más amplia de iniciativas distribuidas.

\paragraph{Conclusión}

Las limitaciones expuestas permiten identificar una línea de trabajo futura enfocada en la mejora de los clusters colaborativos, la cual podría abordarse en dos ejes complementarios:

\begin{itemize}
    \item \textbf{Mejoras técnicas:} desarrollar mecanismos —ya sea mediante extensiones de IPFS Cluster o capas complementarias— que permitan validar de forma efectiva el pinneo distribuido, sin necesidad de replicación total ni dependencia de nodos completamente confiables.
    \item \textbf{Herramientas de participación:} diseñar una plataforma que facilite el despliegue, monitoreo y descubrimiento de aplicaciones comunitarias distribuidas, reduciendo las barreras técnicas y promoviendo la colaboración descentralizada.
\end{itemize}

\subsection{Análisis del consumo de energía en la red de IPFS}

Como se mencionó anteriormente, no hay análisis detallado del uso de energía en la red de IPFS. Un posible trabajo involucra calcular una huella de carbono aproximada, el consumo promedio de energía, entre otras métricas. De ello puede desprenderse un análisis comparativo entre diferentes ecosistemas como Blockchain, y cloud hosting tradicional.

\subsection{Blockchain para aplicaciones comunitarias}

En la búsqueda de una blockchain comunitaria nos encontramos con un servicio llamado \textit{Filecoin} \cite{filecoin} el cual provee de incentivos monetarios para aquellos que cedan espacio en sus discos a guardar archivos de otras personas. Es abierto para que cualquier persona pueda unirse a la red peer-to-peer que proveen. Funciona por arriba de IPFS, utiliza tecnologías de Blockchain para los incentivos y dar garantía que los datos están realmente guardados en los nodos. Nuestra propuesta de trabajo futuro es que se cree una tecnología basada en Blockchain que no necesite del incentivo monetario para funcionar y que además permita la ejecución de código.

\subsection{Análisis de Freenet como ecosistema}

Si en un futuro Freenet finalmente sale en una versión estable, creemos que sería un buen ecosistema al cual realizar un análisis como el de este trabajo y hasta más en profundidad con otro tipo de aplicaciones distribuidas y descentralizadas.
