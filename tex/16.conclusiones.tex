\section{Conclusiones}

Luego de todo el análisis realizado podemos ver que cada ecosistema tiene sus ventajas y desventajas. IPFS y Swarm funcionan muy bien para sitios web estáticos. Una gran ventaja de IPFS es que no requiere costos monetarios para el usuario final, como sí sucede en Ethereum y Swarm. Por otro lado, Ethereum resulta útil para aplicaciones como el repositorio de conocimiento y el mensajero en tiempo real pero se puede volver muy costoso según su implementación, además de temas de usabilidad que se ven más evidentes en el mensajero. Debido a su comunidad y amplia documentación, Ethereum presenta una mejor experiencia de usuario que el resto de ecosistemas. Por su parte, Hyphanet se encuentra prácticamente abandonado y su proyecto sucesor Freenet, si bien parece prometer, todavía está en pleno desarrollo. Concluyendo en que lo más beneficioso sea combinar lo mejor de cada ecosistema para aplicaciones como los casos de uso presentados.

El desarrollo de este trabajo nos brindó la oportunidad de explorar estas tecnologías emergentes las cuales, al estar en desarrollo constante, presentaron varios desafíos que tuvimos que afrontar como la falta de documentación o documentación desactualizada (en Hyphanet principalmente), cambios repentinos de la interfaz de ciertas librerías, faltas de herramientas necesarias para el desarrollo o despliegue de algún ecosistema (lo visto en IPFS) o incluso tecnologías que todavía se encuentran en desarrollo y no tienen una versión estable (como nos sucedió con Freenet).

Por último, hoy podemos decir que adquirimos nuevas herramientas y conocimientos que previamente no poseíamos en la profundidad alcanzada en este trabajo y que, sin duda, podremos aplicar en el futuro.













% No hay un claro ganador, ambos ecosistemas tienen sus ventajas y desventajas. Lo mejor es combinar las ventajas de cada ecosistema.
% (ejemplo de combinación, ipfs para storage y ethereum para ejecución de código).

% Son tecnologías que están en constante cambio. Nos encontramos con varios desafíos: falta de documentación, cambios de API, falta de herramientas en algún ecosistema, etc. La más madura de las analizadas es ethereum.

% Fue un desafío porque no teniamos experiencia previa en estas tecnologías específicas. Es un conocimiento adquirido que podemos usarlo en un futuro. 

% (Por qué creemos que desarrollar aplicaciones descentralizadas puede ser una buena alternativa)