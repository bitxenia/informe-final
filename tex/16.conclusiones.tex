\section{Conclusiones}

Luego de todo el análisis realizado podemos ver que cada ecosistema tiene sus ventajas y desventajas. IPFS y Swarm por su naturaleza como almacenamientos descentralizados funcionan muy bien para sitios web estáticos. En particular, IPFS tiene la gran ventaja de no requerir costos monetarios para el usuario final, si no que promueve la colaboración a través de la contribución de almacenamiento. Por otro lado, Ethereum y Swarm sí requieren un costo monetario para el usuario final pero aseguran una disponibilidad y persistencia mucho mayor. Ethereum resulta útil para aplicaciones como el repositorio de conocimiento y el mensajero en tiempo real pero se puede volver muy costoso según su implementación, y posee algunas cuestiones de usabilidad que son más evidentes en el mensajero. Además, debido a su comunidad y amplia documentación, Ethereum presenta una mejor experiencia de usuario que el resto de ecosistemas. Por su parte, Hyphanet se encuentra prácticamente abandonado y su proyecto sucesor Freenet, si bien parece prometedor, todavía está en pleno desarrollo. Concluyendo en que lo más beneficioso sea combinar lo mejor de cada ecosistema para aplicaciones como los casos de uso presentados.

El desarrollo de este trabajo nos brindó la oportunidad de explorar estas tecnologías emergentes las cuales, al estar en desarrollo constante, presentaron varios desafíos que tuvimos que afrontar como la ausencia de documentación o documentación desactualizada, cambios repentinos de la interfaz de ciertas librerías, faltas de herramientas necesarias para el desarrollo o despliegue de algún ecosistema (lo visto en IPFS), o incluso tecnologías que todavía se encuentran en desarrollo y no tienen una versión estable (como nos sucedió con Freenet). Por otra parte, logramos aplicar los conocimientos técnicos adquiridos en nuestra formación académica y nos adentramos en la comunidad \textit{open source}.

Por último, podemos decir que adquirimos nuevas herramientas y conocimientos que previamente no poseíamos en la profundidad alcanzada en este trabajo y que, sin duda, podremos aplicar en el futuro.
