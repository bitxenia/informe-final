\section{Solución propuesta}

Realizaremos la implementación de 3 casos de uso sobre los distintos ecosistemas a analizar. Presentaremos su código fuente, así como instrucciones para su deploy en las distintas plataformas. Adicionalmente presentaremos un análisis cualitativo y cuantitativo de las ventajas y desventajas de cada caso. A continuación se detalla en qué consiste cada caso de uso.

\subsection{Sitio web informativo}

Este sería el caso más simple, donde no se requiere ningún tipo de procesamiento. El contenido que tendrá este sitio será información sobre los casos de uso implementados, como también instrucciones o referencias de sus ecosistemas de despliegue. Será desarrollado con tecnologías puramente de front end y se experimentará con las herramientas de ambientes distribuidos y descentralizados. Esta página se irá completando a medida que avancemos en el trabajo.

\subsubsection{Requisitos funcionales}

\begin{itemize}
    \item \textbf{Landing page del proyecto}: sitio web informativo donde se presente lo que se fue haciendo en este proyecto, información sobre los casos de uso y sus ecosistemas de despliegue.
\end{itemize}

\subsection{Repositorio de conocimiento}

Con este caso estaríamos analizando la capacidad de creación y modificación de contenido existente. La idea es que sea un servicio comunitario donde agregar información de distinta índole, similar a Wikipedia. Se podrán crear usuarios con roles de moderador y editor. Los moderadores tendrán mayor poder de decisión sobre qué contenido publicar y qué no, además de poder bloquear usuarios de ser necesario. Los editores se encargarán de agregar, eliminar y modificar el contenido del sitio buscando tener la información lo más actualizada posible. Este servicio contará con un front end, back end y una base de datos distribuidas utilizando la tecnología OrbitDB.

\subsubsection{Requisitos funcionales}

\begin{itemize}
    \item \textbf{Edición:} los artículos dentro del repositorio deben ser editables por cualquier persona que ingrese al sitio, y este cambio debe verse reflejado (eventualmente) en las demás personas que accedan a ese artículo.
    \item \textbf{Historial de versiones:} cada artículo debe tener una lista de versiones anteriores, junto con hipervínculos con los cuáles acceder a ellas (mientras estén disponibles en la red).
    \item \textbf{Búsqueda:} una persona debe poder realizar una búsqueda global de todos los artículos.
\end{itemize}

\subsection{Mensajero en tiempo real}

Este caso se enfoca en la capacidad de la infraestructura de enfrentarse a situaciones de \textit{tiempo real} como puede ser un chat de texto o de audio. En particular nos centraremos en el caso de chats de texto para un grupo de usuarios en donde los mensajes sean públicos. Este servicio también contará con front end, back end y una base de datos donde persistir los mensajes.

\subsubsection{Requisitos funcionales}

\begin{itemize}
    \item \textbf{Usuarios:} se deben contar con usuarios que puedan iniciar sesión con una contraseña.
    \item \textbf{Grupos públicos:} grupos de chat de texto, donde cualquier usuario puede ingresar y ver los mensajes del resto, así como también participar enviando sus propios mensajes.
\end{itemize}
