\documentclass{estilo}
\usepackage[spanish]{babel}
\usepackage{graphicx}
\usepackage{float}
\usepackage{amsmath}        % para los vectores columnas
\usepackage{amsfonts}       % para las negrita de pizarra
\usepackage{amssymb}        % para simbolos matematicos
\usepackage{hyperref}       % para utilizar referencias
\usepackage{multirow}       % para las tablas
\usepackage{dsfont}
\usepackage{listings}
\usepackage{xcolor}
\definecolor{codegreen}{rgb}{0,0.6,0}
\definecolor{codegray}{rgb}{0.5,0.5,0.5}
\definecolor{codepurple}{rgb}{0.58,0,0.82}
\definecolor{backcolour}{rgb}{0.95,0.95,0.92}
\lstdefinestyle{mystyle}{
    backgroundcolor=\color{backcolour},   
    commentstyle=\color{codegreen},
    keywordstyle=\color{magenta},
    numberstyle=\tiny\color{codegray},
    stringstyle=\color{codepurple},
    basicstyle=\ttfamily\footnotesize,
    breakatwhitespace=false,         
    breaklines=true,                 
    captionpos=b,                    
    keepspaces=true,                 
    numbers=left,                    
    numbersep=5pt,                  
    showspaces=false,                
    showstringspaces=false,
    showtabs=false,                  
    tabsize=2
}
\lstset{style=mystyle}

\usepackage{enumitem,multicol,setspace}
\newcounter{subenum}[enumi] % para las multicolumnas
\renewcommand{\thesubenum}{\arabic{subenum}}
\usepackage[nomessages]{fp}
\FPeval\thecolwidth{round(1/4:4)}% Specify number of columns -> column width
\newcommand{\newitem}[1]{%
  \refstepcounter{subenum}%
  \parbox{\dimexpr\thecolwidth\linewidth-.5\columnsep}{%
    \makebox[\labelwidth][r]{(\thesubenum)\hspace*{\labelsep}}%
    #1}\hfill%
}

\usepackage{scalerel,stackengine} % para el sombrero
\stackMath
\newcommand\rhat[1]{%
\savestack{\tmpbox}{\stretchto{%
  \scaleto{%
    \scalerel*[\widthof{\ensuremath{#1}}]{\kern-.6pt\bigwedge\kern-.6pt}%
    {\rule[-\textheight/2]{1ex}{\textheight}}%WIDTH-LIMITED BIG WEDGE
  }{\textheight}% 
}{0.5ex}}%
\stackon[1pt]{#1}{\tmpbox}%
}
\parskip 1ex

\usepackage{mathtools}      % floor y ceil
\DeclarePairedDelimiter\ceil{\lceil}{\rceil}
\DeclarePairedDelimiter\floor{\lfloor}{\rfloor}

\usepackage{csquotes}
\usepackage[style=apa, citestyle=numeric, backend=biber,]{biblatex}
\DeclareFieldFormat{labelnumberwidth}{\mkbibbrackets{#1}}
\defbibenvironment{bibliography}
  {\list
     {\printtext[labelnumberwidth]{%
      \printfield{labelprefix}%
      \printfield{labelnumber}}}
     {\setlength{\labelwidth}{\labelnumberwidth}%
      \setlength{\leftmargin}{\labelwidth}%
      \setlength{\labelsep}{\biblabelsep}%
      \addtolength{\leftmargin}{\labelsep}%
      \setlength{\itemsep}{\bibitemsep}%
      \setlength{\parsep}{\bibparsep}}%
      \renewcommand*{\makelabel}[1]{\hss##1}}
  {\endlist}
  {\item}
\addbibresource{citas.bib} %Imports bibliography file


\begin{document}
\maketitle

\tableofcontents

\justifying{}

\newpage
\section{Resumen}

En el siguiente trabajo se analizan distintos ecosistemas y tecnologías que se pueden utilizar para el despliegue de aplicaciones web comunitarias de manera distribuida y descentralizada.

Esto se logra mediante tres casos de uso que ilustran diferentes características: un sitio web informativo, un repositorio de conocimiento y un mensajero en tiempo real. Se comparan ventajas y desventajas del despliegue de cada uno de ellos en IPFS, blockchain y Hyphanet/Freenet, así como también se documenta el proceso del mismo.
\section{Palabras Clave}

Comunitario. Distribuido. Descentralizado. IPFS. Blockchain. Ethereum. Sistema. Aplicación. Peer-to-peer.
\section{Abstract}

The following work analyzes the ecosystems of IPFS, blockchain, and Hyphanet/Freenet, along with their tooling used to develop and deploy community applications in a distributed and decentralized manner.

This is achieved through the implementation of three use cases that illustrate different features: a static website, a knowledge repository, and a real-time messenger, from which a qualitative and quantitative analysis of each ecosystem is conducted.
\section{Keywords}

Community. Distributed. Decentralized. IPFS. Blockchain. Ethereum. System. Application. Peer-to-peer.
\section{Introducción}

Hoy en día, al querer desplegar una aplicación o sitio web comunitario, lo más común es hacerlo a través de un servicio de alojamiento (AWS, Azure, Google Cloud, entre otros) por la comodidad y facilidad que estas ofrecen, alquilando sus servidores para guardar y procesar datos.

Esto puede llegar a traer problemas para este tipo de aplicaciones. Uno de estos problemas puede ser monetario, ya que muchas veces estas aplicaciones dependen de donaciones o voluntarios para sustentarse, como es el caso de Wikipedia. Como también puede suceder que se encuentre en una zona de censura, lo cual facilita su bloqueo al ser servicios centralizados; entre otros problemas más.

Sin embargo, existen otros ecosistemas alternativos que se asemejan mucho más a la filosofía de estas aplicaciones, y que ayudan a combatir estos problemas. En donde las aplicaciones pueden estar alojadas por sus propios usuarios, donando su computo o espacio, y así logrando una descentralización.

En el siguiente documento presentamos un análisis sobre la infraestructura existente, donde es posible la implementación de plataformas para el despliegue de este tipo de aplicaciones, recabando las bondades y desventajas que cada una tiene. 

Para esto se crearon diferentes casos de uso que representan posibles aplicaciones sobre esta metodología alternativa analizando su viabilidad. Entre ellos, se encuentran un sitio web estático, una enciclopedia colaborativa y una aplicación de comunicación en tiempo real.
\section{Estado del Arte}


\section{Problema detectado y/o faltante}

Las soluciones de infraestructura actuales, como AWS, Azure o Google Cloud, presentan barreras significativas para su adopción por parte de proyectos con recursos limitados, como iniciativas independientes, educativas o comunitarias. Estas barreras están relacionadas principalmente con los costos operativos, la dependencia de infraestructura centralizada y la vulnerabilidad frente a fallas o interrupciones del servicio.

\subsection{Costo y sostenibilidad}

La mayoría de las aplicaciones con requerimientos de alta disponibilidad dependen de servicios centralizados en la nube, los cuales implican costos mensuales elevados incluso en etapas tempranas del desarrollo. Este modelo económico desalienta la creación y mantenimiento de servicios comunitarios o de bajo presupuesto, restringiendo su escalabilidad o continuidad en el tiempo.

A su vez, la dependencia de servidores centralizados genera un punto único de mantenimiento y financiamiento que puede convertirse en un cuello de botella. En escenarios donde no existe un respaldo institucional o comercial fuerte, la sostenibilidad del servicio queda comprometida.

\subsection{Interrupciones e infraestructura crítica}

La arquitectura tradicional basada en cliente-servidor implica una fuerte dependencia de la disponibilidad continua de uno o varios nodos centrales. Estas arquitecturas son susceptibles a interrupciones por mantenimiento, errores de configuración, fallas de \textit{hardware} o problemas de conectividad. En muchos casos, un único incidente puede volver una aplicación completamente inaccesible para todos sus usuarios.

Esto evidencia la necesidad de diseñar infraestructuras más resistentes a fallas, donde la continuidad operativa no dependa de un único punto de control.

\subsection{Centralización y control de acceso}

La centralización también facilita el control externo sobre los servicios. Aplicaciones y plataformas pueden ser bloqueadas o restringidas en determinados contextos geográficos o políticos simplemente mediante la identificación y bloqueo de sus puntos de acceso. Esto representa una amenaza para la disponibilidad global y el acceso libre a herramientas comunitarias.

Un ejemplo representativo es el de Wikipedia, cuyo acceso ha sido bloqueado —de forma total o parcial— en diversos países, debido a restricciones impuestas sobre determinados contenidos \parencite{censorship-wikipedia}.

\subsection{Accesibilidad tecnológica}

Finalmente, muchas de las tecnologías necesarias para implementar infraestructuras distribuidas aún requieren conocimientos técnicos avanzados para su configuración, despliegue y operación. Esta complejidad técnica representa una barrera de entrada tanto para usuarios como para desarrolladores que podrían beneficiarse de este tipo de arquitectura, pero que no cuentan con los recursos o conocimientos necesarios para adoptarla.
\section{Solución propuesta}


\section{Metodología aplicada}

La metodología aplicada para la gestión del proyecto fue una versión aproximada a Scrum. El desarrollo se dividió en sprints semanales para los cuales utilizamos un tablero Kanban en Github Projects donde se fueron agregando las tareas a realizar para cada caso de uso y ecosistema. Se realizaron reuniones semanales fijas que se usaron como punto de control, donde se revisó lo hecho durante la semana y definimos pasos a seguir para las siguientes. También nos fue útil para detectar posibles ajustes o cambios de rumbo que fueron surgiendo a lo largo del trabajo. Al finalizar cada una de estas reuniones realizamos minutas que nos sirvieron para resumir lo tratado en cada una y tener en claro los pasos a seguir después de las mismas.

Durante el descubrimiento de las funcionalidades de cada caso de uso realizamos \textit{User Story Mappings} (USM) para organizar las tareas de realizar.

Los distintos artefactos que fueron surgiendo durante el desarrollo del trabajo fueron almacenados en un Google Drive compartido entre todo el equipo. Dentro del mismo se pueden encontrar: minutas de reuniones, cronogramas, USM, entre otros.

Para el control de versiones del código se creó una organización en Github en la que se fueron creando repositorios para los distintos paquetes que integraron el trabajo.

La modalidad fue virtual y asincrónica. Nos mantuvimos en constante comunicación a través de un servidor de Discord y también se realizaron sesiones de \textit{pair} y \textit{mob-programming} en distintas ocasiones.
\section{Experimentación y/o validación}

\subsection{Costos}
¿Cuánto nos cuesta desplegar y mantener un servicio en cada ecosistema?

\subsubsection{IPFS}

\subsubsection{Blockchain}


De los casos de uso esperamos responder las siguientes incógnitas: %obtener las siguientes métricas:

\paragraph{Swarm}
Al deployar el sitio web es necesario contar con *postage stamps* que son la manera de pagar por el uso del almacenamiento en Swarm. Cada actualización que se realice al sitio requiere de *postage stamps* y, además, estos tienen fecha de vencimiento por lo que es necesario volver a pagar frecuentemente. Hay que tener en cuenta que dichos *postage stamps* se pagan en la criptomoneda BZZ que fluctúa de valor con respecto al dólar estadounidense.

La obtención del sitio web no requiere de costo alguno, por lo que desde el punto de vista de un usuario de la aplicación no sería necesario pagar.

<TODO: medir cuánto es el costo aproximado en USD o BZZ>

\paragraph{Ethereum}
Se utiliza la moneda ETH para pagar por el despliegue de cada transacción, esto incluye tanto el despliegue de cada *smart contract* como también la edición de un artículo (en el caso del repositorio de conocimiento). Por lo tanto, el usuario final de la aplicación termina pagando por creación y edición de cada artículo en el repositorio de conocimiento, y por cada mensaje enviado en el mensajero en tiempo real. Por otro lado, para las operaciones de lectura no se tiene que pagar nada.

<TODO: medir cuánto es el costo aproximado en USD o ETH>

\subsection{Facilidad de desarrollo}

¿Qué tan fácil es desplegar en cada ecosistema?

\subsubsection{IPFS}

\subsubsection{Blockchain}

\paragraph{Swarm}
En Swarm existe la herramienta de terminal [`swarm-cli`](https://github.com/ethersphere/swarm-cli) con la cual se puede interactuar con un nodo de Swarm. También el equipo de Swarm provee una *Github Action* que permite la posibilidad de automatizar el despliegue generando un pipeline que utilice dicha herramienta.

En cuanto a un ambiente de pruebas o staging, si bien no existe un gateway público que interactúe con la *testnet*, es posible levantar uno propio que sí lo haga apuntando a la *testnet* de Sepolia usando la herramienta [`gateway-proxy`](https://github.com/ethersphere/gateway-proxy).

\paragraph{Ethereum}
Con la librería web3.js se puede interactuar con un nodo de Ethereum y realizar un despliegue de la aplicación. Además, existen las herramientas de Hardhat con las cuales se puede levantar una red de prueba que facilita el desarrollo local.

\subsection{Viabilidad} 

¿Que tan viable es crear una aplicación comunitaria para cada uno de estos ecosistemas?

\subsubsection{IPFS}

\subsubsection{Blockchain}

\paragraph{Swarm}
Resulta más conveniente para sitios web o recursos estáticos. No es posible la ejecución de código.

\paragraph{Ethereum}
Su punto fuerte es la ejecución de código, por lo cual es útil para funcionar como backend para aplicaciones web. Por el costo de almacenamiento de los smart contracts no es recomendable para sitios o recursos estáticos como imágenes o videos.

\subsection{Performance}

\subsubsection{IPFS}

\subsubsection{Blockchain}

\subsection{Resumen}

% Poner la tabla, hay una en el notion 
\section{Plan de actividades}


\section{Referencias}

\printbibliography[heading=none]
\newpage
\end{document}